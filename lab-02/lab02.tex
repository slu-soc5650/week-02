% set document atributes, margins, and paper size:
\documentclass[letterpaper,11pt]{article} 
\usepackage[margin=1in]{geometry}
% ============================================================

% define the title
\title{Introduction to Geographic Information Science \\ \large Lab 02}
\author{Christopher G. Prener, Ph.D.}
\date{Spring, 2016}

% ============================================================
\begin{document}
% ============================================================
\maketitle % generates the title
% ============================================================
\section*{Directions}
If you have not already done so, download the week-02 repository from GitHub. The lab-02 folder contains the datasets needed to complete this lab exercise. Begin by opening the .do file editor window. As you complete each question below, copy the working code into your .do file. Each command should be given its own separate line. Each time you add a line of code, add a corresponding comment that describes what you did in your own words. Try to organize your .do file in a way that reflects the order of the questions answered. Email your .do file to me when it is complete.

\section*{Lab Exercises}
\subsection*{Review from Week 1}
1. Open the \texttt{citytemp.dta} file built into Stata's memory \\
2. Describe the variables in this dataset \\
3. Use the \texttt{describe} command to describe only three variables at once (pick any three) \\
4. Summarize the variable \texttt{heatdd} \\

\subsection*{Working Directories}
5. Clear Stata's memory \\
6. Display your current working directory - add a note to your .do file with the file path \\
7. Change your working directory to the folder you downloaded from GitHub named lab-02 \\

\subsection*{Using and Saving Stata Files}
8. Open the Stata dataset file saved in lab-02 \\
9. Open the data viewer to quickly explore your data (include only a note in your .do file saying you did this)\\
10. Save the Stata dataset file in a way that overwrites the existing data\\
11. Save the Stata dataset file again, this time with a new file name\\
12. Describe the variables in this dataset\\
13. Produce descriptive statistics for the variable \texttt{pop} - in the .do file, include a description of the average given for this variable.\\

\subsection*{Using CSV files}
14. Clear Stata's memory\\
15. Import the CSV file saved in lab-02\\
16. Open the data viewer to quickly explore your data (include only a note in your .do file saying you did this)\\

\subsection*{Using Excel files}
17. Clear Stata's memory\\
18. Import the excel file saved in lab-02\\
19. Open the data viewer to quickly explore your data (include only a note in your .do file saying you did this)\\

\subsection*{Working with File Paths}
20. Clear Stata's memory\\
21. Re-open the Stata dataset file \texttt{census.dta}, this time by specifying its full file path\\
22. Clear Stata's memory\\

% ============================================================
\end{document}